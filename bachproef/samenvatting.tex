%%=============================================================================
%% Samenvatting
%%=============================================================================

% TODO: De "abstract" of samenvatting is een kernachtige (~ 1 blz. voor een
% thesis) synthese van het document.
%
% Een goede abstract biedt een kernachtig antwoord op volgende vragen:
%
% 1. Waarover gaat de bachelorproef?
% 2. Waarom heb je er over geschreven?
% 3. Hoe heb je het onderzoek uitgevoerd?
% 4. Wat waren de resultaten? Wat blijkt uit je onderzoek?
% 5. Wat betekenen je resultaten? Wat is de relevantie voor het werkveld?
%
% Daarom bestaat een abstract uit volgende componenten:
%
% - inleiding + kaderen thema
% - probleemstelling
% - (centrale) onderzoeksvraag
% - onderzoeksdoelstelling
% - methodologie
% - resultaten (beperk tot de belangrijkste, relevant voor de onderzoeksvraag)
% - conclusies, aanbevelingen, beperkingen
%
% LET OP! Een samenvatting is GEEN voorwoord!

%%---------- Nederlandse samenvatting -----------------------------------------
%
% TODO: Als je je bachelorproef in het Engels schrijft, moet je eerst een
% Nederlandse samenvatting invoegen. Haal daarvoor onderstaande code uit
% commentaar.
% Wie zijn bachelorproef in het Nederlands schrijft, kan dit negeren, de inhoud
% wordt niet in het document ingevoegd.

\IfLanguageName{english}{%
\selectlanguage{dutch}
\chapter*{Samenvatting}

\selectlanguage{english}
}{}

%%---------- Samenvatting -----------------------------------------------------
% De samenvatting in de hoofdtaal van het document

\chapter*{\IfLanguageName{dutch}{Samenvatting}{Abstract}}

Tegenwoordig is er een exponentiële groei in het gebruik van e-commerce webshops. De onderliggende database speelt hierbij een belangrijke rol. Om developers binnen bedrijven in de toekomst te kunnen ondersteunen bij hun keuze voor een databaseoplossing, zal er een onderzoek gevoerd worden naar de meest geschikte database oplossing voor een e-commerce webshop in de mode-industrie. Het onderzoek richt zich op het evalueren van vijf verschillende database oplossingen. De focus zal hierbij gelegd worden op de prestaties en kosten van de desbetreffende databases. Het onderzoek zelf zal onderverdeeld worden in zes verschillende fases. In eerste instantie zal er een literatuurstudie worden gevoerd. In de stand van zaken zal informatie worden gegeven over de verschillende soorten databases die er bestaan en zal nagegaan worden welke databases momenteel gebruikt worden door het bedrijf Aware. Na de literatuurstudie volgt een requirements analyse waarin de functionele- en niet-functionele requirements voor dergelijk databasesysteem besproken worden. In een volgende stap wordt een short list van vijf verschillende databases opgesteld. Vervolgens wordt een vergelijkende studie van deze vijf databases uitgevoerd. Deze studie bestaat enkel uit gevonden literatuur en technische gegevens van deze databases. Nadien volgt een Proof of Concept (PoC) waarin elk van deze databases uitgetest wordt. In laatste instantie wordt op basis van de vergelijkende studie en de resultaten van de PoC een besluit getrokken die weergegeven wordt in de conclusie.

\vspace{5mm}

Op basis van dit onderzoek kunnen we besluiten dat de keuze van de database deels afhankelijk is van de voorkeuren van de developer. Indien geopteerd wordt voor een SQL databasesysteem wordt ClickHouse aangeraden. Indien de developer een voorkeur heeft voor een NoSQL database wordt vooral MongoDB aangeraden.