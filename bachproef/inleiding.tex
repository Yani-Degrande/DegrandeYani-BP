%%=============================================================================
%% Inleiding
%%=============================================================================

\chapter{\IfLanguageName{dutch}{Inleiding}{Introduction}}%
\label{ch:inleiding}

Tegenwoordig speelt e-commerce een steeds grotere rol in ons dagelijks leven. Dit wordt onder meer gestimuleerd door de opkomst van influencers die hun merken online willen promoten. Hiervoor is het belangrijk dat webshops in staat zijn om efficiënt om te gaan met aanzienlijke hoeveelheden dataverkeer. De onderliggende database is hierbij van cruciaal belang. Deze moet in staat zijn om goed te kunnen presteren wanneer er omgegaan wordt met grote hoeveelheden data. Idealiter is deze database ook kostenefficiënt om de bedrijfskosten te minimaliseren.
De juiste keuze van een database is belangrijk omdat deze de steunpilaar vormt van een modern e-commerce systeem. De database zorgt voor het opslaan en verwerken van klantengegevens en productinformatie maar ondersteunt ook onderliggende transacties en gebruikersinstructies. Het is daarom van belang dat deze database hoge prestaties kan leveren onder verschillende omstandigheden zoals tijdens piekverkoopperiodes in bijvoorbeeld de solden periode.
Naast de hoge prestaties die een database moet kunnen leveren is de kostprijs ook een belangrijke factor. De totale kosten van een database systeem omvatten niet enkel de initiële setup en licentiekosten, als die er zijn, maar ook de operationele kosten gerelateerd aan het onderhoud, de schaalbaarheid en de mogelijke aanpassingen die noodzakelijk zijn om te voldoen aan de veranderende bedrijfsbehoeften.
Het doel van deze literatuurstudie is om een selectie aan database oplossingen te vergelijken op basis van hun prestaties en kosten. Er zal een onderscheid gemaakt worden tussen de verschillende types database (NoSQL, SQL, ...) om de voor- en nadelen van elke soort te onderzoeken. Achteraf zullen vijf databases besproken en geëvalueerd worden om te bepalen welke het beste past bij de gegeven use case.


% \begin{itemize}
%   \item context, achtergrond
%   \item afbakenen van het onderwerp
%   \item verantwoording van het onderwerp, methodologie
%   \item probleemstelling
%   \item onderzoeksdoelstelling
%   \item onderzoeksvraag
%   \item \ldots
% \end{itemize}

\section{\IfLanguageName{dutch}{Probleemstelling}{Problem Statement}}%
\label{sec:probleemstelling}

Het is algemeen bekend dat e-commerce een steeds grotere rol speelt in ons dagelijks leven.
Meer en meer bedrijven en organisaties maken gebruik van webshops om hun producten te verkopen.
Deze webshops vergen echter een goede database om efficiënt te kunnen werken.
De database moet in staat zijn om goed te kunnen presteren wanneer er omgegaan wordt met grote hoeveelheden data en moet de groeiende hoeveelheden aan data kunnen verwerken.
Voor developers binnen bedrijven of organisaties die zich bezighouden met e-commerce is het echter niet altijd eenvoudig om de juiste database te kiezen.
Er zijn namelijk verschillende soorten databases die elk hun eigen voor- en nadelen hebben.
Deze bachelorproef richt zich specifiek op de mode-industrie maar kan ook toegepast worden op andere sectoren.



% Uit je probleemstelling moet duidelijk zijn dat je onderzoek een meerwaarde heeft voor een concrete doelgroep. De doelgroep moet goed gedefinieerd en afgelijnd zijn. Doelgroepen als ``bedrijven,'' ``KMO's'', systeembeheerders, enz.~zijn nog te vaag. Als je een lijstje kan maken van de personen/organisaties die een meerwaarde zullen vinden in deze bachelorproef (dit is eigenlijk je steekproefkader), dan is dat een indicatie dat de doelgroep goed gedefinieerd is. Dit kan een enkel bedrijf zijn of zelfs één persoon (je co-promotor/opdrachtgever).

\section{\IfLanguageName{dutch}{Onderzoeksvraag}{Research question}}%
\label{sec:onderzoeksvraag}

Wat is de beste database oplossing voor een e-commerce webshop in de mode-industrie met een focus op de prestaties en prijs van dergelijke database. Hierbij zal er een onderscheid gemaakt worden tussen de verschillende types database (NoSQL, SQL, ...) om de voor- en nadelen van elke soort te onderzoeken. Achteraf zullen vijf verschillende databases besproken en geëvalueerd worden om te bepalen welke het beste past bij de gegeven use case.

% Wees zo concreet mogelijk bij het formuleren van je onderzoeksvraag. Een onderzoeksvraag is trouwens iets waar nog niemand op dit moment een antwoord heeft (voor zover je kan nagaan). Het opzoeken van bestaande informatie (bv. ``welke tools bestaan er voor deze toepassing?'') is dus geen onderzoeksvraag. Je kan de onderzoeksvraag verder specifiëren in deelvragen. Bv.~als je onderzoek gaat over performantiemetingen, dan

\section{\IfLanguageName{dutch}{Onderzoeksdoelstelling}{Research objective}}%
\label{sec:onderzoeksdoelstelling}

Uit een combinatie van een vergelijkende studie, waarbij vooraf opgestelde functionele en niet-functionele vereisten worden gehanteerd, en een testopstelling moet blijken welke database uit de shortlist het beste presteert en tegelijkertijd kostenefficiënt genoeg is voor een e-commerce webshop in de mode-industrie.

\section{\IfLanguageName{dutch}{Opzet van deze bachelorproef}{Structure of this bachelor thesis}}%
\label{sec:opzet-bachelorproef}

% Het is gebruikelijk aan het einde van de inleiding een overzicht te
% geven van de opbouw van de rest van de tekst. Deze sectie bevat al een aanzet
% die je kan aanvullen/aanpassen in functie van je eigen tekst.

De rest van deze bachelorproef is als volgt opgebouwd:

Hoofdstuk~\ref{ch:stand-van-zaken} geeft een overzicht van de stand van zaken binnen het onderzoeksdomein, op basis van een literatuurstudie.

Hoofdstuk~\ref{ch:methodologie} licht de methodologie toe en vermeldt de gebruikte onderzoekstechnieken die gebruikt worden om een antwoord te kunnen formuleren op de onderzoeksvragen.

% TODO: Vul hier aan voor je eigen hoofstukken, één of twee zinnen per hoofdstuk
In Hoofdstuk~\ref{ch:requirement-analyse} wordt een analyse gemaakt van de functionele en niet-functionele vereisten voor een database in een e-commerce webshop in de mode-industrie.

In Hoofdstuk~\ref{ch:shortlist} wordt een short list opgesteld van de vijf verschillende databases die vergeleken gaan worden.

In Hoofdstuk~\ref{ch:vergelijkende-studie} bevindt zich de vergelijkende studie. Deze studie is gebaseerd op de technische gegevens van de verschillende databases op de shortlist.

In Hoofdstuk~\ref{ch:onderzoek} worden verschillende testen uitgevoerd op de verschillende databases uit de short list. Hierin worden ook de bevindingen geanalyseerd en besproken.

In Hoofdstuk~\ref{ch:conclusie}, tenslotte, wordt de conclusie gegeven en een antwoord geformuleerd op de onderzoeksvragen. Daarbij wordt ook een aanzet gegeven voor toekomstig onderzoek binnen dit domein.