%%=============================================================================
%% Conclusie
%%=============================================================================

\chapter{Conclusie}%
\label{ch:conclusie}

% TODO: Trek een duidelijke conclusie, in de vorm van een antwoord op de
% onderzoeksvra(a)g(en). Wat was jouw bijdrage aan het onderzoeksdomein en
% hoe biedt dit meerwaarde aan het vakgebied/doelgroep? 
% Reflecteer kritisch over het resultaat. In Engelse teksten wordt deze sectie
% ``Discussion'' genoemd. Had je deze uitkomst verwacht? Zijn er zaken die nog
% niet duidelijk zijn?
% Heeft het onderzoek geleid tot nieuwe vragen die uitnodigen tot verder 
%onderzoek?

In de vergelijkende studie werd tot de conclusie gekomen dat het best gebruik wordt gemaakt van een NoSQL database voor het opzetten van een e-commerce webshop in de mode-industrie. Na het uitvoeren van enkele testen op vier verschillende databases: MariaDB, MongoDB, Couchbase en ClikHouse werd echter vastgesteld dat Clickhouse mogelijks de beste oplossing is van de vier. Aangezien het belangrijk is dat zowel de bevindingen van de vergelijkende studie als de testresultaten in beschouwing worden genomen kan hier geconcludeerd worden dat er meerdere databases aangeraden kunnen worden voor een e-commerce webshop in de mode-industrie. Hoewel ClickHouse betere prestaties vertoonde tijdens de testen is het geen NoSQL database maar een column-oriented SQL database management system. De resultaten van de testen verschillen echter niet veel met deze van de NoSQL aanhanger MongoDB. Zowel MongoDB als ClickHouse zouden beiden een goede optie kunnen zijn voor het opzetten van een e-commerce webshop in de mode-industrie. 

\vspace{5mm}

Alvorens een database te kiezen is het van belang de vraag te stellen welke gegevens opgeslagen moeten worden en op welke wijze dit moet gebeuren. Als het bedrijf dat een e-commerce webshop aanmaakt een SQL aanpak verkiest, is ClickHouse een goede oplossing. Als geopteerd wordt voor een NoSQL aanpak zijn MongoDB of Couchbase een aangewezen oplossing. Hoewel de resultaten voor het ophalen van gegevens tijdens de testen minder gunstig waren voor Couchbase in vergelijking met deze van MongoDB mag Couchbase niet dadelijk afgeschreven worden. De mindere resultaten kunnen immers het gevolg zijn van onder andere de configuratie van de database zelf. Naast het kiezen van de juiste database is het natuurlijk ook belangrijk om na te gaan hoe de database het best geoptimaliseerd kan worden voor de use case. Dit onderwerp werd in deze bachelorproef niet behandeld maar kan wel interessant zijn voor een volgend onderzoek.  