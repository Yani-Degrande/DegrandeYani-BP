%%=============================================================================
%% Voorwoord
%%=============================================================================

\chapter*{\IfLanguageName{dutch}{Woord vooraf}{Preface}}%
\label{ch:voorwoord}

%% TODO:
%% Het voorwoord is het enige deel van de bachelorproef waar je vanuit je
%% eigen standpunt (``ik-vorm'') mag schrijven. Je kan hier bv. motiveren
%% waarom jij het onderwerp wil bespreken.
%% Vergeet ook niet te bedanken wie je geholpen/gesteund/... heeft

De zoektocht naar een geschikt onderwerp voor mijn bachelorproef toegepaste informatica was moeilijker dan verwacht.
Ik was op zoek naar een onderwerp dat aansloot bij mijn interesses alsook bij mijn specialisatie binnen de opleiding toegepaste informatica.
Zo kwam ik al vlug uit op de categorie web en app development.
Websites en applicaties ontwikkelen is meer dan gewoon een passie geworden voor mij. Een passie die ik graag verder wil ontwikkelen.
Daarom leek het me ook logisch om een onderwerp te kiezen binnen deze categorie.

Dit was echter niet zo eenvoudig.
Vele onderwerpen binnen de categorie web en app development waren al reeds uitgebreid onderzocht en beschreven.

Als ontspanning bekijk ik graag een video op youtube.
Dit herinnerde me aan het feit dat vele influencers en youtubers hun eigen merchandise verkopen via een webshop.
Influencers en youtubers zijn echter niet de enige personen die gebruik maken van webshops.
Er zijn steeds meer bedrijven en organisaties die gebruik maken van webshops om hun producten te verkopen.

Het opzetten en beheren van een goeddraaiende webshop is echter veel uitgebreider dan het doet lijken.
Een van de belangrijkste aspecten van een webshop is de database.
De database is het hart van de webshop, de plaats waar alle data wordt opgeslagen.
Net om deze reden is de keuze van de gepaste database van cruciaal belang.
Hierdoor kwam ik op het idee om een onderzoek te doen naar de beste database oplossing voor het opzetten van een webshop in de mode-industrie.
Op deze manier wil ik proberen om toekomstige webshop eigenaars te helpen bij het maken van de juiste keuze omtrent de database van hun webshop.

Ik wil graag mijn promotor, Sonia Vandermeersch, alsook mijn co-promotor, Sergio Braet, bedanken voor de begeleiding en feedback die ik kreeg tijdens het schrijven van deze bachelorproef.