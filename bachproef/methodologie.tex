%%=============================================================================
%% Methodologie
%%=============================================================================

\chapter{\IfLanguageName{dutch}{Methodologie}{Methodology}}%
\label{ch:methodologie}

%% TODO: In dit hoofstuk geef je een korte toelichting over hoe je te werk bent
%% gegaan. Verdeel je onderzoek in grote fasen, en licht in elke fase toe wat
%% de doelstelling was, welke deliverables daar uit gekomen zijn, en welke
%% onderzoeksmethoden je daarbij toegepast hebt. Verantwoord waarom je
%% op deze manier te werk gegaan bent.
%% 
%% Voorbeelden van zulke fasen zijn: literatuurstudie, opstellen van een
%% requirements-analyse, opstellen long-list (bij vergelijkende studie),
%% selectie van geschikte tools (bij vergelijkende studie, "short-list"),
%% opzetten testopstelling/PoC, uitvoeren testen en verzamelen
%% van resultaten, analyse van resultaten, ...
%%
%% !!!!! LET OP !!!!!
%%
%% Het is uitdrukkelijk NIET de bedoeling dat je het grootste deel van de corpus
%% van je bachelorproef in dit hoofstuk verwerkt! Dit hoofdstuk is eerder een
%% kort overzicht van je plan van aanpak.
%%
%% Maak voor elke fase (behalve het literatuuronderzoek) een NIEUW HOOFDSTUK aan
%% en geef het een gepaste titel.

Om een gestructureerd overzicht te bieden van de bachelorproef, is het onderzoek opgedeeld in zes verschillende fasen. Hieronder volgt een beknopte toelichting van elke fase, inclusief de doelstellingen en methoden die worden gebruikt om de doelen van die fase te bereiken.

\section{\IfLanguageName{dutch}{Fase 1: Literatuurstudie}{Phase 1: Literature Review}}%
\label{sec:literatuurstudie-methodologie}

De eerste fase van het onderzoek is de stand van zaken. Hierin wordt meer informatie verzameld over de huidige stand van zaken op vlak van de gebruikte databaseoplossingen binnen Aware. In de stand van zaken wordt ook meer gedetailleerde informatie verstrekt over de verschillende soorten databases en hun voor- en nadelen. Dit onderzoek is gebeurd op basis van de meest relevante academische papers en technische documentatie. De stand van zaken dient ook een beter beeld te scheppen over de scope van de bachelorproef.

\textbf{Deliverables:} Stand van zaken.

\section{\IfLanguageName{dutch}{Fase 2: Requirement-analyse}{Phase 2: Requirement-analysis}}%
\label{sec:analysis}

In de daaropvolgende fase wordt er een requirement analyse opgesteld. In deze fase worden de functionele en niet-functionele vereisten voor een e-commerce webshop in de mode-industrie geanalyseerd. Hiervoor wordt er gebruik gemaakt van verschillende relevante bronnen omtrent het onderwerp. 

\textbf{Deliverables:} Een requirement analyse.

\section{\IfLanguageName{dutch}{Fase 3: Opstellen short-list}{Phase 3: Making a short-list}}%
\label{sec:short-list}

De fase die volgt op de requirement analyse omvat het opstellen van een shortlist aan onderzoekswaardige databases. Aangezien het onmogelijk is om elke bestaande database te vergelijken binnen de opgegeven tijd voor deze bachelorproef wordt eerst een longlist aan verschillende bestaande databases bekeken die nadien omgezet worden in een short list. Voor het opstellen van de short list wordt ook rekening gehouden met de huidige databaseoplossing gebruikt door Aware. Deze zullen de basis vormen voor de short list. Naast de huidige gebruikte oplossingen worden drie andere databases besproken. Elk van deze databases heeft zijn voor- en nadelen en is geschikt voor het gebruik in een e-commerce webshop in de mode-industrie. Later in de bachelorproef wordt bekeken welke van de vijf verschillende databases het meest geschikt is voor een webshop. 

\textbf{Deliverables:} Een short list aan databases die vergeleken moeten worden.

\section{\IfLanguageName{dutch}{Fase 4: Vergelijkende studie}{Phase 4: Comparative Study}}%
\label{sec:studie}

Na het opstellen van een short list volgt een fase waarin de vijf verschillende databases vergeleken worden met elkaar op basis van de gevonden documentatie voor elk van deze vijf databases. Deze vergelijking zal een algemeen beeld geven van welke databases in theorie beter zouden moeten presteren dan andere. 

\textbf{Deliverables:} Een conclusie over welke database best geschikt is voor een e-commerce webshop in de mode-industrie, gebaseerd op technische documentatie.

\section{\IfLanguageName{dutch}{Fase 5: Testopstelling/PoC}{Phase 5: Test Setup/PoC}}%
\label{sec:test}

In fase vijf volgen kleine testen waarin de prestaties (zoals de querysnelheid) uitgevoerd worden om de vergelijking verder te zetten. Deze testen zijn echter niet representatief ten opzichte van een echte webshop. Tijdens de testen worden gelijkaardige queries uitgevoerd maar op een relatief kleinere hoeveelheid data. Niet enkel dit maar ook andere factoren zoals de omgeving waarin de queries uitgevoerd worden kunnen een vertekend beeld geven. Het resultaat van de vergelijking mag dus niet enkel afhangen van deze testen.

\textbf{Deliverables:} Een conclusie over welke database best geschikt is voor een e-commerce webshop in de mode-industrie, gebaseerd op technische testen.

\section{\IfLanguageName{dutch}{Fase 6: Conclusie}{Phase 6: Conclusion}}%
\label{sec:conclusie}

Als laatste wordt de conclusie uitgeschreven. Hierin worden de bevindingen van zowel de vergelijkende studie en de testen beschreven. Daarnaast wordt ook een conclusie getrokken om een antwoord te kunnen formuleren op de onderzoeksvraag.

\textbf{Deliverables:} Een antwoord op de onderzoeksvraag.
